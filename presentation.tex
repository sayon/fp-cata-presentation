% !TEX TS-program = pdflatex
% !TEX encoding = UTF-8 Unicode

\documentclass{beamer}


\mode<presentation>
{
  %\usetheme{Warsaw}

%  \setbeamercovered{transparent}
  % or whatever (possibly just delete it)
}
\setbeamertemplate{frametitle}
  {\begin{centering}\smallskip
   \insertframetitle\par
   \smallskip\end{centering}}
\setbeamertemplate{itemize item}{$\bullet$}
\setbeamertemplate{navigation symbols}{}
\setbeamertemplate{footline}[text line]{%
    \hfill\strut{%
        \scriptsize\sf\color{black!60}%
        \quad\insertframenumber
    }%
    \hfill
}

\usepackage[english]{babel}
% or whatever

\usepackage[utf8]{inputenc}
% or whatever		

\usepackage{times}
\usepackage[T1]{fontenc}


\usepackage{mathrsfs}
\usepackage{mathtools}
\usepackage{proof} 


%% Coq listings need it: %%
\definecolor{ltblue}{rgb}{0,0.4,0.4}
\definecolor{dkblue}{rgb}{0,0.05,0.4}
\definecolor{dkgreen}{rgb}{0,0.35,0}
\definecolor{dkviolet}{rgb}{0.3,0,0.5}
\definecolor{dkred}{rgb}{0.5,0,0}
\definecolor{grey}{rgb}{0.5,0.5,0.6}
\usepackage{listings}
\usepackage{lstcoq}
\lstset{
basicstyle=\footnotesize,
columns=fullflexible, 
language=Coq} 

%settings 
\everymath{\displaystyle}

%--------


\title[and its variations]{Calculus of constructions explained}
 
\author{I.Zhirkov}
 

\date{2015}
\subject{Theoretical Computer Science}


% Delete this, if you do not want the table of contents to pop up at
% the beginning of each subsection:
\AtBeginSubsection[]
{
  \begin{frame}<beamer>{Outline}
    \tableofcontents[currentsection,currentsubsection]
  \end{frame}
}


% If you wish to uncover everything in a step-wise fashion, uncomment
% the following command: 

%\beamerdefaultoverlayspecification{<+->}


\begin{document}

\begin{frame}
  \titlepage
\end{frame}

\begin{frame}{Outline}
  \tableofcontents
\end{frame}

\section{Syntax}
\begin{frame}
\frametitle{What is CC?}
\begin{itemize}
\item A higher-order formalism  for constructive proofs in natural deduction style;
\pause
\item Informally: typed $\lambda$-calculus, where types are normal terms too;
\item Or: high level functional programming language with a type system rich enough to express any specification.
\end{itemize}
\end{frame}


 
\subsection{Basic syntax}
\begin{frame}{Basic CC syntax}
\begin{itemize}
\item Variables (are typed with other terms).
\item Application:

$M \: N$
\item $\lambda$-abstraction:

$ (\lambda x : T ) M $

\item Product operator

$ [x: T] M$

think \textit{forall} or \textit{function type}.
\item Special constant \textit{*}.

\end{itemize}
\end{frame}


\begin{frame}{About *}
\begin{itemize}
\item ``Type of all types'';
\item Type of propositions;
\item Products over * are special (are called \textit{contexts});
\item Has no type itself.
\end{itemize}
\end{frame}

\subsection{Contexts and objects}

\begin{frame}{Contexts}
We will call some of terms \textbf{contexts}.

Contexts are lists.

\begin{itemize}
\item Nil = *
\item Cons = Product operator
\end{itemize} 

$$ * \in C^n$$

$$ [x:M] N \in C^n, M \in \Lambda^n,  N \in C^{n+1}$$

$x$ may occur in $N$, not in $M$

Contexts have no type (cause * has none).

Contexts are type of proposition schemas (think nat->bool->Prop), declaring their parameters.

\end{frame}

\begin{frame}{Terms can be}
\begin{itemize}
\item Contexts (products over *);
\item Objects;
\pause
\begin{itemize}
\item Objects of type * are \textbf	{usual types} aka \textbf{propositions};
\item Other objects.
\end{itemize}
\end{itemize}

\pause

$$\Lambda = C \cup O$$
$$\Lambda^n = C^n \cup O^n$$
$$O = \bigcup_{n\geq 0}  O^n, \quad C = \bigcup_{n\geq 0}  C^n$$

$[\text{variables' de Bruijn indices}] \leq n$

$\Lambda$ denotes all terms;

$C$ or $\Lambda_o$ -- contexts;

$O$ or $\Lambda_l$ -- objects.

\end{frame}



\begin{frame}{Objects}

\begin{itemize}
\item Variables (think de Bruijn indexes)

\begin{center}
$ k \in O^n$, if $1 \leq k \leq n$
\end{center}
\item Product

\begin{center}
$ [x : M ] N \in O^n$, if  $M \in \Lambda^n$, $N \in O^{n+1} $
\end{center}

\item Abstraction
\begin{center}
$ ( \lambda x : M) N \in O^n $, if $ M \in \Lambda^n$, $N \in O^{n+1}$
\end{center}

\item Application
\begin{center}
$(M \; N) \in O^n$, if $M, N \in O^n$
\end{center}

\end{itemize}
\end{frame}


\section{Interpretation}
\begin{frame}{Two relations, two kinds of judgements}
\begin{itemize}

\item $\Gamma \vdash \Delta$

$\Delta$ \textit{is valid in a valid context } $\Gamma$


\item $\Gamma \vdash M : N$

In valid  $\Gamma$, $M$ is well-typed of type $N$

\end{itemize}

\end{frame}

\begin{frame}{Inference}

Contexts:

$$\infer{\vdash *}{}$$

$$\infer{\Gamma [ x : \Delta ] \vdash *}{\Gamma \vdash \Delta}$$

$$\infer{\Gamma [ x : M ] \vdash *}{\Gamma \vdash M : *}$$

\pause
Product:

$$\infer{\Gamma \vdash [x:M] \Delta}{\Gamma [x:M] \vdash \Delta}$$

$$\infer{\Gamma \vdash [x:M_1] M_2 : * }{\Gamma [x:M_1] \vdash M_2 : *}$$
\end{frame}


\begin{frame}{Inference}
\begin{itemize}
\item Variables ($ l \leq | \Gamma | $)

$$\infer{\Gamma \vdash l : \Gamma / l} {\Gamma \vdash *}$$

\item Abstraction
$$\infer{\Gamma \vdash (\lambda x : M_1) M_2 : [x:M_1] P}{\Gamma [x:M_1] \vdash M_2 : P}$$

\item Application

$$\infer{\Gamma \vdash (M N ) : [N/x]Q}{\Gamma \vdash M : [x:P] Q & \Gamma \vdash N:P}$$

\textit{this is substitution $N \mapsto 1$,being var idx}

\end{itemize}
\end{frame}


\begin{frame}{Note on predicativity}

\textit{A \textbf{predicative} system enforces the constraint that, when an object is defined using some sort of quantifier, none of the quantifiers may ever be instantiated with the object itself.} 

Avoid naive set theory paradoxes. 

\begin{itemize}
\pause
\item Martin-Loef type theory has a hierarchy of type universes (predicative):

$$( [A: U_i] A ) : U_{i+1}$$

\pause
\item Calculus of constructions has only one universe $*$ and is nonpredicative:

$$( [A: *] A ) : *$$

$*$ has no type however.

%\textit{In nonpredicative CoC inductive entities can be defined.}
\end{itemize}

\end{frame}



\begin{frame}{Note on predicativity}

\begin{itemize}
\item Adding $\infer{* \vdash * : *}{}$ results in Girard paradox (each type is inhabited).
\end{itemize}

  
\end{frame}



\subsection{Conversion rules}

\begin{frame}{Third kind of judgements}

\begin{itemize}

\item $\Gamma \vdash \Delta$

$\Delta$ \textit{is valid in a valid context } $\Gamma$


\item $\Gamma \vdash M : N$

\textit{In valid } $\Gamma$, $M$ is well-typed of type $N$


\item $\Gamma \vdash M \cong N$

\textit{$M$ is convertible to $N$ in $\Gamma$.}

Smallest congruence over terms w.r.t. $\beta$ reduction. 

\end{itemize}
\end{frame}

\begin{frame}{Kinds of calculus of constructions}
\begin{itemize}
\item Restricted -- $\beta$-reduction only on 'logical' level (in types)
\item Full -- $\beta$-reduction on both type level and term level is allowed.
\end{itemize}


Restricted's only use is a simplier proof of decidability of three judgements.
\end{frame}


\subsubsection{Restricted CC}
\begin{frame}{$\beta$-reduction rules for restricted CoC}
Symmetry, transitivity of $\cong$;

\textit{$\Delta$ is a context}

$$\infer{\Gamma \vdash \Delta \cong \Delta}{\Gamma \vdash \Delta}$$

\pause
$$\infer{\Gamma \vdash M \cong M}{\Gamma \vdash M: \Delta}$$  
\pause
$$\infer{\Gamma \vdash [x:P_1] \; M_1 \cong [x : P_2] M_2} {\Gamma \vdash P_1 \cong P_2 & \quad \Gamma [x: P_1] \vdash M_1 \cong M_2 } $$
 
\pause
$$\infer{\Gamma \vdash (\lambda x:P_1) M_1 \cong (\lambda x : P_2) M_2} {\Gamma \vdash P_1 \cong P_2 & \quad \Gamma [x: P_1] \vdash M_1 \cong M_2 & \quad \Gamma [x: P_1] \vdash M_1 : \Delta_1} $$
 

\end{frame}

\begin{frame}{$\beta$-reduction rules for restricted CoC - 2}
 
 
$$\infer{\Gamma \vdash (M \; N) \cong (M_1 \; N)} {\Gamma \vdash (M \; N) : \Delta & \quad \Gamma \vdash M \cong M_1} $$

$$\infer{\Gamma \vdash (M \; N) \cong (M \; N_1)}{\Gamma \vdash (M \; N) : \Delta & \quad \Gamma \vdash N \cong N_1}$$

$$\infer{\Gamma \vdash ((\lambda x : P) M \; N) \cong [N/x] M}{\Gamma[x:P]\vdash M:\Delta & \quad \Gamma \vdash N: P} $$

 $\beta$-reduction only for ``logical'' redexes.

 
$$\infer{\Gamma \vdash M : Q}{\Gamma \vdash M:P & \quad \Gamma \vdash P \cong Q}$$


For restricted CoC decidability of judgements is proven independently from normalization theorem. 

\end{frame}


\subsubsection{Full CC}

\begin{frame}{$\beta$-reduction rules for full CoC-1}
Symmetry, transitivity of $\cong$;


$$\infer{\Gamma \vdash M \cong M }{\Gamma \vdash M: N}$$  


$$\infer{\Gamma \vdash [x:P_1] \; M_1 \cong [x : P_2] M_2} {\Gamma \vdash P_1 \cong P_2 & \quad \Gamma [x: P_1] \vdash M_1 \cong M_2 } $$
 
\pause
$$\infer{\Gamma \vdash (\lambda x:P_1) M_1 \cong (\lambda x : P_2) M_2} {\Gamma \vdash P_1 \cong P_2 & \quad \Gamma [x: P_1] \vdash M_1 \cong M_2 & \quad \Gamma [x: P_1] \vdash M_1 : N_1} $$


\end{frame}

\begin{frame}{$\beta$-reduction rules for full CoC-2}

$$\infer{\Gamma \vdash (M \; N) \cong (M_1 \; N_1)} {\Gamma \vdash (M \; N) : P & \quad \Gamma \vdash M \cong M_1 & \quad \Gamma \vdash N \cong N_1} $$

$$\infer{\Gamma \vdash ((\lambda x : A) M \; N) \cong [N/x] M}{\Gamma[x:A] \vdash M:P & \quad \Gamma \vdash N: A}$$

\end{frame}


\begin{frame}{Comparison-1}
\begin{itemize}
\item Restricted
$$\infer{\Gamma \vdash M \cong M}{\Gamma \vdash M: \Delta}$$  

\item Full
$$\infer{\Gamma \vdash M \cong M }{\Gamma \vdash M: N}$$  


\end{itemize}
\end{frame}


\begin{frame}{Comparison-2}
\begin{itemize}
\item Restricted
$$\infer{\Gamma \vdash (\lambda x:P_1) M_1 \cong (\lambda x : P_2) M_2} {\Gamma \vdash P_1 \cong P_2 & \quad \Gamma [x: P_1] \vdash M_1 \cong M_2 & \quad \Gamma [x: P_1] \vdash M_1 : \Delta_1} $$

\item Full
$$\infer{\Gamma \vdash (\lambda x:P_1) M_1 \cong (\lambda x : P_2) M_2} {\Gamma \vdash P_1 \cong P_2 & \quad \Gamma [x: P_1] \vdash M_1 \cong M_2 & \quad \Gamma [x: P_1] \vdash M_1 : N_1} $$


\end{itemize}
\end{frame}


\begin{frame}{Comparison-3}
\begin{itemize}
\item Restricted
$$\infer{\Gamma \vdash (M \; N) \cong (M_1 \; N)} {\Gamma \vdash (M \; N) : \Delta & \quad \Gamma \vdash M \cong M_1} $$

$$\infer{\Gamma \vdash (M \; N) \cong (M \; N_1)}{\Gamma \vdash (M \; N) : \Delta & \quad \Gamma \vdash N \cong N_1}$$

\item Full
$$\infer{\Gamma \vdash (M \; N) \cong (M_1 \; N_1)} {\Gamma \vdash (M \; N) : P & \quad \Gamma \vdash M \cong M_1 & \quad \Gamma \vdash N \cong N_1} $$

\end{itemize}
\end{frame}


\begin{frame}{Comparison-4}
\begin{itemize}
\item Restricted
$$\infer{\Gamma \vdash ((\lambda x : P) M \; N) \cong [N/x] M}{\Gamma[x:P]\vdash M:\Delta & \quad \Gamma \vdash N: P}$$
 
\item Full
$$\infer{\Gamma \vdash ((\lambda x : A) M \; N) \cong [N/x] M}{\Gamma[x:A] \vdash M:P & \quad \Gamma \vdash N: A}$$
 
\end{itemize}
\end{frame}


\subsection{Stripping}


\begin{frame}{Stripping}
\begin{itemize}
\item We can get rid of specification and extract untyped $\lambda$-term, representing computation.
\item Contexts get rid of quantifications over contexts;

Why? A context is a type of proposition schemas, its inhabitants are not part of computation process.

%$$j^* = Id_0$$
%$$\Gamma = \Delta [x: \text{Ctx} ] \Rightarrow  j_{\Gamma}(k) = j_{\Delta}(k) + 1$$
%$$\Gamma = \Delta [x: \text{Obj} ] \Rightarrow  j_{\Gamma}(k+1) = j_{\Delta}(k) + 1,\quad j_{\Gamma}(1) = 1$$

\end{itemize}
\end{frame}
  

\begin{frame}[fragile]{Stripping for $\Gamma \vdash M : Obj$}
\textbf{Context arity} $\alpha_{\Gamma}$ -- how many elements of list are objects;

\textbf{Canonical injection} $j_{\Gamma} : \{1 \dots \alpha_{\Gamma} \} \rightarrow \Gamma$ -- indices of objects in $\Gamma$. 

\textbf{Stripped algorithm} $\nu_{\Gamma}(M)$: 

\begin{itemize}
\item Variable: 

$\nu_{\Gamma}(k) \coloneqq j_{\Gamma}^{-1}(k)$
\item Application:

$\nu_{\Gamma}( (M:\_) (N: \text{Ctx}) ) \coloneqq  \nu_{\Gamma}(M)$

$\nu_{\Gamma}( (M:\_) (N: \text{Obj}) ) \coloneqq  \nu_{\Gamma}(M)\; \nu_{\Gamma}(N)$

\item Abstraction:

$\begin{cases}
\nu_{\Gamma}( (\lambda x: \text{Obj}_1 \;  N: \text{Obj}_2 ) \coloneqq \lambda x. \nu_{\Delta}( N ) , \quad & \Delta = \Gamma [x: \text{Obj}_1] \\
\nu_{\Gamma}( (\lambda x: \text{Ctx} \; \;  N: \text{Obj} ) \coloneqq \nu_{\Delta}( N ) , \quad & \Delta = \Gamma [x: \text{Ctx}] 
\end{cases} $ 


\end{itemize}



\end{frame}

\subsection{Constructions}

\begin{frame}{Interpretation of constructions}
\begin{itemize}
\item Take all closed terms, add special constant:

$$\mathscr{I} = \Lambda^0 \cup \{ \Omega \}$$

\item $A \subset \mathscr{I}$ is \textbf{saturated} iff:

\begin{enumerate}
\item $\Omega \in A$
\item $SN(b_1) \dots SN(b_n)  \Rightarrow (\Omega b_1 \dots b_n) \in A$
\item all elements in $A$ are $SN$
\item $SN(b) \Rightarrow (([ b/x ] a) b_1 \dots b_n) \in A \Rightarrow (((\lambda x. a) b) b_1\dots b_n) \in A$
\end{enumerate}
\item $\mathscr{U}$ -- all saturated subsets of $\mathscr{I}$
\end{itemize}
\textit{SN} denotes \textit{strongly normalisable}


\end{frame}


\begin{frame}[fragile]{Dependent product}

\begin{definition}
If $A \in \mathscr{U}$ , $F \in \mathscr{I} \rightarrow \mathscr{U}$, then 
$\Pi(A,F) \coloneqq \{ t \in \mathscr{I} | \forall x \in A : \; (t \; x) \in F(x)\}$

\end{definition}

~
 
$\mathscr{I}$ are programs, $\mathscr{U}$ are types, $\mathscr{I}\rightarrow \mathscr{U}$ are dependent types.

\begin{lemma}
$\mathscr{U}$ is closed under arbitrary intersection and union.

\end{lemma}

$\Omega$ is only needed during proof. 


\end{frame}


\begin{frame}{Functionality}
\begin{definition}[restricted calculus]
$\phi(\mbox{Object}) \coloneqq  \mathscr{I} $

$\phi(*) \coloneqq \mathscr{U}$

$\phi([x:P]\Gamma) = \phi(P) \rightarrow \phi(\Gamma) \quad\quad \mbox{(set of all such functions)}$

\end{definition}


\begin{definition}[full calculus]
$\phi(\mbox{Object}) \coloneqq  \mathscr{I} $

$\phi(*) \coloneqq \mathscr{U}$

$\phi([x:\Delta]\Gamma) = \phi(\Delta) \rightarrow \phi(\Gamma)$

$\phi([x:P]\Gamma) = \{ f \in \phi(P) \rightarrow \phi(\Gamma)  | t \cong u \Rightarrow f(t) = f(u) \}$


Why? cause arbitrary $\beta$-reduction is allowed.

\end{definition}
\end{frame}

\begin{frame}{Environment}
\begin{definition} 
For $\Gamma = [x_n: A_n] \cdots [x_1: A_1]*$ \textbf{environment} is a product:

$\Phi(\Gamma) = \phi(A_n) \times \dots \times \phi(A_1)$
\end{definition}

\end{frame}

\begin{frame}{$\Gamma \vdash M : \text{Obj}$}
Interpretation: $\rho_{\Gamma}(M): \Phi(\Gamma) \rightarrow\phi(\text{Obj})$


$\nu_{\Gamma}(M)$ is an $\alpha_{\Gamma}$-ary function $\mathscr{I} ^{\alpha_{\Gamma}} \rightarrow \mathscr{I}$ (maps arguments to its $\alpha_{\Gamma}$ free variables).

recall that $\alpha_{\Gamma}$ is context arity. 



$\pi_{\Gamma} : \Phi(\Gamma) \rightarrow \mathscr{I}^{\alpha_{\Gamma}}$ is defined by type forgetting $j_{\Gamma}.$

Interpretation: $\rho_{\Gamma}(M) = \nu_{\Gamma}(M) \circ \pi_{\Gamma}$

``untype $M$ and forget about quantifications over contexts''
\end{frame}

\begin{frame}{$\Gamma \vdash M : \text{Ctx} $}

$\rho_{\Gamma}(M): \Phi(\Gamma) \rightarrow\phi(\text{Ctx})$

\begin{itemize}
\item Product:
\begin{itemize}
\item $\Gamma \vdash [x: \text{Ctx}] M : * $

$f: \Phi(\Gamma)\times \phi(\text{Ctx})  \rightarrow \mathscr{U}$

$f \coloneqq \rho_{\Gamma [x: \text{Ctx}]}(M)$

$$\rho_{\Gamma}([x: \text{Ctx}] M) \coloneqq a \mapsto \bigcap \{ f(a,x) | x \in \phi(\text{Ctx}) \}, \quad a \in \Phi(\Gamma)$$

\item $\Gamma \vdash [x: \text{Obj}] M : * $


$f : \Phi(\Gamma)\rightarrow \mathscr{U}, \quad \, \, \, \quad g : \Phi(\Gamma) \times \mathscr{I} \rightarrow \mathscr{U}$
$f \coloneqq \rho_{\Gamma[x:\text{Obj}]}(\text{Obj}), \quad g \coloneqq \rho_{\Gamma[x:\text{Obj}]}(M)$

$$\rho_{\Gamma}([x: \text{Obj}] M) \coloneq a \mapsto \Pi(f(a), g(a)), \quad a \in \Phi(\Gamma)$$


\end{itemize}
\end{itemize}


\end{frame}

\begin{frame}{$\Gamma \vdash M : \text{Ctx} $}

$\rho_{\Gamma}(M): \Phi(\Gamma) \rightarrow\phi(\text{Ctx})$

\begin{itemize}
\item Variable: $\rho_{\Gamma}(x)$ selects $x$-th variable from context.
\item Abstraction: 

$\Gamma \vdash (\lambda x: M) \; N : [x:M]P$

$f : \Phi(\Gamma) \times \phi(M) \rightarrow \phi(P)$

$f \coloneqq \rho_{\Gamma[x:M]}(N)$

$$\rho_{\Gamma} ((\lambda x: M) N) \coloneqq a \mapsto \bigg(x \mapsto f(a,x)\bigg)$$

\item Application

$\Gamma \vdash (M \; N) : [N/x] P$

$$\rho_{\Gamma} ((M \; N)) \coloneqq  y \mapsto \rho_{\Gamma}(M)(y, \rho_{\Gamma}(N,y))$$

\end{itemize}

\end{frame}

\begin{frame}{Interpretation of contexts}
Inductively defined on formation of $\Gamma \vdash *$  (routine).

$D(\Gamma) \xhookrightarrow{} \Phi(\Gamma)$

\begin{itemize}
\item $\vdash *$ implies $D(\Gamma) = \Phi(\Gamma) = 1$

\item $\Gamma[x:\text{Obj}] \vdash * $ implies $D(\Gamma[x:\text{Obj}]) \coloneqq \{(a,x) | a \in D(\Gamma) \land x \in \rho_{\Gamma}(M)(a)\}$

\item $\Gamma[x:\text{Ctx}] \vdash * $ implies $D(\Gamma[x:\text{Ctx}]) \coloneqq \{(a,x) | a \in D(\Gamma) \land x \in \phi(\text{Ctx}))\} $ 

$= D(\Gamma) \times \phi(\text{Ctx})$
\end{itemize} 
\begin{example}
$[A: *][x:A] \vdash * $ is interpreted as $\{(A,x) \in \mathscr{U} \times \mathscr{I} | x\in A\}$.
\end{example}

\end{frame}

\begin{frame}

    
    If $\Gamma \vdash M: N$ and $\Gamma \vdash N : *$, then for all $x \in D(\Gamma)$ pure $\lambda$-term $\rho_{\Gamma}(M,x)$ is an element of saturated set $\rho_{\Gamma}(P,x)$  

    \begin{example}

If $[A:*][x:A] \vdash x:A$ then with $\Gamma = [A:*][x:A]$ we have $D(\Gamma)  = \{ (A,x) \in \mathscr{U} \times \mathscr{I} | x \in A \}$.

$[A:* ] [x:A] \Gamma \vdash x:A$ is interpreted as  $f: \mathscr{U} \times \mathscr{I} \rightarrow \mathscr{I}$, which maps $(A,x) \mapsto x$.
    \end{example}


\end{frame}
\subsection{Properties}

\begin{frame}
\begin{itemize}
\item 
\item $\Gamma \vdash Judgement$ is decidable (restricted);
\item $\Gamma \vdash M:N$ deducing $N$ from $\Gamma$ and $M$ is decidable(restricted, full);
\item Strongly normalizable (restricted,full).
\item $\Gamma \vdash M \cong N \Rightarrow \phi(M) = \phi(N)$
\end{itemize}
\end{frame}



\section{Modifications}
\begin{frame}{CoC modifications}
\begin{itemize}
\item Calculus of Constructions;
\item Calculus of Constructions with Inductive Definitions;
\item Coq v7: Calculus of (Co)Inductive constructions (Cic);
\item Coq v8: Predicative Calculus of (Co)Inductive constructions (pCic)
\end{itemize}
\end{frame}
\subsection{Utility tweaks}
\begin{frame}{Utility tweaks}
\begin{itemize}
\item Global environement;
\item Typing rules consider environement;
\item Additional reductions for \textit{let... in...} constructs and global definitions;
\end{itemize}
\end{frame}


\subsection{Universes, impredicativity}
\begin{frame}[fragile]{Universes and impredicativity}
\begin{itemize}
\item Coq has many type universes (sorts): \textit{Set}, \textit{Prop}, \textit{Type}$_i$ for $i \in \mathbb{N}$
\item $Set$ : $Type_0$, $Prop$ : $Type_0$, $Type_0$ : $Type_1$ etc.
\item Since Coq v8 \textit{Set} is predicative by default (unless launched with \texttt{-impredicative-set}), so such definitions

\begin{lstlisting} 
Definition nat : Set := forall (C:Set), C -> (C->C) ->C.
\end{lstlisting}

are not allowed.
\end{itemize}
\end{frame}
\subsection{Inductive definitions}
\begin{frame}[fragile]{Inductive definitions}
In nonpredicative CoC inductive types are expressed as non first-class objects.

\begin{lstlisting}
Definition nat : Set := forall (C:Set), C -> (C -> C) -> C.
Definition zero : nat := fun C z f => z.
Definition succ : nat -> nat := fun n C z f => f (n C z f).
\end{lstlisting}

\end{frame}

%\section{CIC  Coq}
%\subsection{On match}

\end{document}


