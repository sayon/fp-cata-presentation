% !TEX TS-program = pdflatex
% !TEX encoding = UTF-8 Unicode

\documentclass{beamer}


\mode<presentation> { }


\usepackage[english]{babel} 

\usepackage[utf8]{inputenc} 
\usepackage{xcolor}
\usepackage{listings}
\usepackage{lsthaskell}
\usepackage{times}
\usepackage[T1]{fontenc}
\usepackage{verbatim}
\usepackage{mathrsfs}
\usepackage{mathtools}
\usepackage{proof} 
\usepackage{amscd}
\setbeamertemplate{frametitle}
  {\begin{centering}\smallskip
   \insertframetitle\par
   \smallskip\end{centering}}
\setbeamertemplate{itemize item}{$\bullet$}
\setbeamertemplate{navigation symbols}{}
\setbeamertemplate{footline}[text line]{%
    \hfill\strut{%
        \scriptsize\sf\color{black!60}%
        \quad\insertframenumber
    }%
    \hfill
}

\definecolor{ltblue}{rgb}{0,0.4,0.4}
\definecolor{dkblue}{rgb}{0,0.05,0.4}
\definecolor{dkgreen}{rgb}{0,0.35,0}
\definecolor{dkviolet}{rgb}{0.3,0,0.5}
\definecolor{dkred}{rgb}{0.5,0,0}
\definecolor{grey}{rgb}{0.5,0.5,0.6}
\usepackage{listings}
\lstset{
basicstyle=\footnotesize,
columns=fullflexible, 
language=Haskell} 

%settings 
\everymath{\displaystyle}

%--------


\title[]{Different recursion schemas and their properties}
 
\author{I.Zhirkov}
 

\date{2015}
\subject{Theoretical Computer Science}

\newcommand{\hin}[2]{\lstinputlisting[firstline=#1, lastline=#2]{listings/algebra.hs}}

% Delete this, if you do not want the table of contents to pop up at
% the beginning of each subsection:
\AtBeginSubsection[]
{
  \begin{frame}<beamer>{Outline}
    \tableofcontents[currentsection,currentsubsection]
  \end{frame}
}


% If you wish to uncover everything in a step-wise fashion, uncomment
% the following command: 

%\beamerdefaultoverlayspecification{<+->}


\begin{document}

\begin{frame}
  \titlepage
\end{frame}

%\begin{frame}{Outline}
%  \tableofcontents
%\end{frame}

 
 
\begin{frame}
\frametitle{Morphisms}
\begin{itemize} 
    \item Catamorphisms (destruction)
    \item Anamorphisms (construction)
    \item Hylomorphisms (combination of two)
    \item Paramorphisms (saves more information than hylomorphisms)
\end{itemize}
\end{frame}


\section{Catamorphisms}

\begin{frame}{Catamorphism}
\begin{itemize}
\item ``generalized fold''

$b::B$



\begin{align*}
h::&\  \text{List}\ A \rightarrow B\\ 
\\
h\ Nil &= b\\
h\ (Cons \ a\ as) &=  f\ a\ (h\ as)
\end{align*}


~

\item Notation: $h = (\!|b, f|\!)$
\item Arises from algebra ($f\ a \rightarrow a$).
\end{itemize}

\end{frame}

\begin{frame}{Anamorphism}
\begin{itemize}
\item ``generalized unfold''

$p::\ B\rightarrow Bool$

$g:: B \rightarrow (A,B)$ 


~

\begin{align*}
h::&\  B \rightarrow List\ A\\ 
h\ b &= Nil, & p \ b \\
h\ b &=  Cons a\ (h\ b'),& otherwise
\end{align*}

where $(a,b') = g\ b $



\item Notation $h = [\!(g,p)\!]$
\item Arises from coalgebra ($a \rightarrow f\ a$).
\end{itemize}

\end{frame}

\begin{frame}{Hylomorphism}
\begin{itemize}
\item ``call-tree looks like data structure''

$c::\ C$

$f:: B\rightarrow C \rightarrow C$ 

$g:: A \rightarrow (B,A)$


$p:\ A\rightarrow Bool$

~

\begin{align*}
h::&\   \rightarrow List\ A\\ 
h\ a &= c, & p \ a \\
h\ b &= f\ b (h\ a'), & otherwise
\end{align*}

where $(b, a') = g\ a$



\item Notation $h = [\![(c,f), (g,p)]\!]$
\end{itemize}

\end{frame}

\begin{frame}{Hylomorphism-2}
\begin{itemize}

\item Is a composition of anamorphism and catamorphism

$[\![(c,f), (g,p)]\!] = (\!|c, f|\!) \circ [\!(g, p)\!]$

\item \textit{Look at the whiteboard for a fancy call-tree image}
\item Example: factorial is $[\![(1, \times), (g,p)]\!]$, where:

\begin{align*}
p\ n &= n == 0 \\
g\ n &= (n, n-1)
\end{align*}
\end{itemize}
\end{frame}


\begin{frame}[fragile]{Paramorphism}

\begin{itemize}
\item Hylomorphism for \textit{fac }is not inductively defined on \textit{nat}.
\item A nat paramorphism example:

\begin{align*}
h\ 0&= b\\
h\ (\text{Succ} \ n ) &= f\ n\ (h\ n)
\end{align*}

\item A list example:

\begin{align*}
h\ \text{Nil}& = b\\
h\ (\text{Cons}\ x\ xs ) &= f\ x\ xs\ (h\ xs)
\end{align*}

\item Notation: $\langle\![ b, f ]\!\rangle$
\end{itemize}

\end{frame}

\begin{frame}{Category}
$C = (obj, hom, \circ)$

Objects and morphisms between objects with their compositions.

Laws:
\begin{itemize}
\item Identity morphisms for each object
\item Composition is associative
\item Path equality
\end{itemize}

``Point'' is synonymous to ``Object''
\end{frame}


\begin{frame}{Categories: $Set$}
\begin{itemize}
\item $ob(Set)$ -- all sets
\item $hom(E,F)$ -- functions between sets $E$ and $F$
\item $\circ$ -- composition
\end{itemize}

\end{frame}


\begin{frame}{Categories: $Set$}
\begin{itemize}
\item $ob(Set)$ -- all sets
\item $hom(E,F)$ -- functions between sets $E$ and $F$
\item $\circ$ -- composition
\item Note: $ob(Set)$ is not a set itself (is a class).
\end{itemize}

\end{frame}

\begin{frame}{Categories}
\begin{itemize}
\item $Mon$: (monoids, morphisms, composition)
\item $Grp$: (groups, morphisms, composition)
\item $Hask$: (haskell types, functions, \hasqel{(.)})
\item ...
\end{itemize}

\end{frame}

\begin{frame}{Functor}
\begin{itemize}
\item Functor is a morphism between categories (preserves structures)
\end{itemize}

\end{frame}
\begin{frame}{Functor in FP}
\begin{itemize}
\item an (endo-)functor is an operation from types to types (from Hask to Hask)
\item preserves identity and composition.
\item functions can be 'mapped over' functors
\item Basic ones: identity, product, sum (tagged), arrow \dots

\end{itemize}

\end{frame}



\begin{frame}[fragile]{A usual recursive datatype}
\hin{9}{11}

is same as:
\hin{16}{20}
\end{frame}


\begin{frame}[fragile]{Test}
\hin{23}{24}
\end{frame}

 
\begin{frame}[fragile]{It is a functor!}

\hin{29}{32}

It is possible to construct an algebra on top of any functor.

\hin{55}{55}

We expect to be able to 'evaluate' children of \hasqel{Expr}.
Example:

\hin{37}{40}

\end{frame}




\begin{frame}[fragile]{What is an algebra?}

$(C, F, A, m)$
\begin{itemize}
\item Category $C$ ($Hask$, points are types of Haskell)
\item Endofunctor $F$ (maps Hask points into other Hask points)
\item Point $A$ (\textbf{carrier}) of category $C$ (some type).
\item Function $m$ mapping $F\ A \rightarrow A$
\end{itemize}
 
Hence our definition:

\hin{55}{55}

$C$ is implied ($Hask$), $F$ and $A$ are type-level, the rest is $m$, the function itself (of type \hasqel{Algebra f a}).

\textit{Look at the whiteboard for a fancy image!}

\end{frame}


\begin{frame}[fragile]{Initial algebra}
\begin{itemize}
\item There are infinitely many algebras based on same functor

\item One is particular \textbf{Initial algebra}, 

\item Does not 'forget' anything, preserves all information about input.

\item $\exists$ a homomorphism from initial algebra to any other algebra.

\item Assume: carrier is \hasqel{Fix f}, algebra is \hasqel{Fx} (its ctor).

\hin{55}{55}
\hin{63}{65}

\end{itemize}

\end{frame}

\begin{frame}[fragile]

Functor $f$ is fixed. Let $a$ be the carrier object for a new algebra.

~

\begin{tabular}{r | c | c}
 ~ & Carrier & Evaluator\\\hline
Initial algebra&   Fix f & Fx\\\hline
Some algebra &     a & alg \\
\end{tabular}

\end{frame}

\begin{frame}[fragile]{Constructing any algebra-1}


We want to get a homomorphism from initial algebra to some other algebra. Carrier mapper:

\hasqel{g:: Fix f -> a}

Remember: \hin{16}{16}


Thanks to the fact that $f$ is a functor, fmap is at our disposal to map : 

\hasqel{fmap g :: f (Fix f) ->  f a}

\begin{center}
$
\begin{CD}
f (\text{Fix} \ f) @>\text{fmap} \ g>> f\ a\\
@VVFxV @VValgV\\
\text{Fix} \ f @>g>> a
\end{CD}
$

\end{center}
\end{frame}


\begin{frame}[fragile]{Constructing any algebra-2}

\hasqel{Fx} is lossless, thus invertible.

\hin{106}{107}

$
\begin{CD}
f (\text{Fix} \ f) @>\text{fmap} \ g>> f\ a\\
@VVFxV @VValgV\\
\text{Fix} \ f @>g>> a
\end{CD}
$
~~~~~becomes 
$
\begin{CD}
f (\text{Fix} \ f) @>\text{fmap} \ g>> f\ a\\
@AAunFixA @VValgV\\
\text{Fix} \ f @>g>> a
\end{CD}
$

~

~

$g$ can be defined with unfix, fmap and evaluator:

\hin{110}{110}
\end{frame}

\begin{frame}[fragile]{Meet catamorphism}

\hin{110}{110}


\hin{115}{116}

Quick check:

\hin{43}{46}

\begin{verbatim}
*Main> :t cata alg
cata alg :: Fix ExprF -> String
\end{verbatim}
\end{frame}


\begin{frame}[fragile]{Coalgebra}

Algebra: $f\ a \rightarrow a$

Coalgebra: $a \rightarrow f\ a$

~

$
\begin{CD}
f (\text{Fix} \ f) @>\text{fmap} \ g>> f\ a\\
@VVFxV @VValgV\\
\text{Fix} \ f @>g>> a
\end{CD}
$
~~~~~becomes 
$
\begin{CD}
f (\text{Fix} \ f) @<\text{fmap} \ g<< f\ a\\
@AAunFixA @AAcoalgA\\
\text{Fix} \ f @<g<< a
\end{CD}
$

~

~

~
The same reasoning about initial coalgebra applies.


\end{frame}


\begin{frame}{Coalgebra-2}
$\mu$ combines Fx and unFix. 

\hin{141}{141}

\hin{142}{142}
\hin{55}{55}
\hin{145}{149}
\end{frame}

\begin{frame}{Coalgebra-3}

As $out$ is an initial algebra, $in$ is a terminal coalgebra (there exists a unique homomorphism from any coalgebra to $in$).
\begin{center}

$
\begin{CD}
f (\mu \ f) @<\text{fmap} \ g<< f\ a\\
@VVinV @AAcoalgA\\
\mu \ f @<g<< a
\end{CD}
$

\end{center}
\end{frame}

\begin{frame}{Notation summary}
Expanding notation to arbitrary types:
\begin{itemize}
\item Catamorphism:

$(\!|\phi|\!)_F = \mu(\phi \xleftarrow{F} out)$ 

\item Anamorphism:

$[\!(\psi)\!]_F = \mu(in \xleftarrow{F} \psi )$ 

\item Anamorphism:

$[\!(\psi)\!]_F = \mu(in \xleftarrow{F} \psi )$ 


\end{itemize}

\end{frame}




\begin{frame}
``Actually, even in Haskell recursion is not completely first class because the compiler does a terrible job of optimizing recursive code. This is why F-algebras and F-coalgebras are pervasive in high-performance Haskell libraries like vector, because they transform recursive code to non-recursive code, and the compiler does an amazing job of optimizing non-recursive code.''

\end{frame}
\begin{frame}
Further reading:
\begin{itemize}
\item Control.Functor.Algebra
\end{itemize}
\end{frame}

\end{document}


